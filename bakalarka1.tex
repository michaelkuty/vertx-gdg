%% LyX 1.5.5 created this file.  For more info, see http://www.lyx.org/.
%% Do not edit unless you really know what you are doing.
\documentclass[a4paper,twoside,czech,czech,openright,cleardoubleempty,BCOR10mm,DIV11]{scrreprt}
\usepackage[T1]{fontenc}
\usepackage[utf8]{inputenc}
\usepackage{array}
\usepackage{longtable}
\usepackage{varioref}
\usepackage{wrapfig}
\usepackage{fancybox}
\usepackage{calc}
\usepackage{framed}
\usepackage{url}
\usepackage{graphicx}

\makeatletter

%%%%%%%%%%%%%%%%%%%%%%%%%%%%%% LyX specific LaTeX commands.
\providecommand{\LyX}{L\kern-.1667em\lower.25em\hbox{Y}\kern-.125emX\@}
\newcommand{\lyxline}[1][1pt]{%
  \par\noindent%
  \rule[.5ex]{\linewidth}{#1}\par}
\newcommand{\noun}[1]{\textsc{#1}}
%% Special footnote code from the package 'stblftnt.sty'
%% Author: Robin Fairbairns -- Last revised Dec 13 1996
\let\SF@@footnote\footnote
\def\footnote{\ifx\protect\@typeset@protect
    \expandafter\SF@@footnote
  \else
    \expandafter\SF@gobble@opt
  \fi
}
\expandafter\def\csname SF@gobble@opt \endcsname{\@ifnextchar[%]
  \SF@gobble@twobracket
  \@gobble
}
\edef\SF@gobble@opt{\noexpand\protect
  \expandafter\noexpand\csname SF@gobble@opt \endcsname}
\def\SF@gobble@twobracket[#1]#2{}
%% Because html converters don't know tabularnewline
\providecommand{\tabularnewline}{\\}

%%%%%%%%%%%%%%%%%%%%%%%%%%%%%% Textclass specific LaTeX commands.
\newenvironment{lyxcode}
{\begin{list}{}{
\setlength{\rightmargin}{\leftmargin}
\setlength{\listparindent}{0pt}% needed for AMS classes
\raggedright
\setlength{\itemsep}{0pt}
\setlength{\parsep}{0pt}
\normalfont\ttfamily}%
 \item[]}
{\end{list}}

%%%%%%%%%%%%%%%%%%%%%%%%%%%%%% User specified LaTeX commands.
%<-------------------------------společná nastavení------------------------------>
\usepackage[czech]{babel}%počeštění názvů (Obsah, Kapitola, Literatura atp.)
\usepackage[]{hyperref} %odkazy v  pdf jsou klikací s barevnými rámečky
\usepackage[numbers,sort&compress]{natbib} %balíček pro citace literatury  
\usepackage{hypernat}%interakce mezi hyperref a natbib
\newcommand{\BibTeX}{{\sc Bib}\TeX}%BibTeX logo
\hypersetup{   % Nastavení polí PDF dokumentu 
pdftitle={Sablona pro psani zaverecnych praci v LyXu},%   
pdfauthor={Vitezslav Vydra},%  
pdfsubject={},%   
pdfkeywords={\v{s}ablona,LaTeX,LyX}%                             
}
\usepackage{multicol}




%<-----------------------------volání stylů----------------------------------------->
% (znak % je označení komentáře: co je za ním, není aktivní)
%<------------------------------------písmo----------------------------------------->
%\usepackage{packages/bc-latinmodern}
%\usepackage{packages/bc-times}
\usepackage{packages/bc-palatino}
%\usepackage{packages/bc-iwona}
%\usepackage{packages/bc-helvetika}


%<------------------------------záhlaví stránek------------------------------------>
%\usepackage{packages/bc-headings}
\usepackage{packages/bc-fancyhdr}

%<------------------------------hlavičky kapitol------------------------------------>
%\usepackage{packages/bc-neueskapitel}
\usepackage{packages/bc-fancychap}

\makeatother

\usepackage{babel}

\begin{document}
~\thispagestyle{empty}{\small ~\vfill{}
}{\small \par}

\noindent {\small Na tomto místě mohou být napsána případná poděkování
(vedoucímu práce, konzultantovi, tomu kdo půjčil software, literaturu,
poskytl data apod.). \newpage{}}{\small \par}

~\thispagestyle{empty}\vfill{}
Tato stránka je tzv. protititul a je graficky součástí titulní stránky.
Nechte ji prázdnou, nebo na ni umístěte vhodnou fotografii či ilustraci.

\cleardoublepage{}~\thispagestyle{empty}\begin{center}\pagenumbering{roman}\vspace{10mm}


\textsf{\textsc{\noun{\LARGE České vysoké učení technické v Praze}}}\\
\vspace{0.5em}
\textsf{\textsc{\noun{\LARGE Fakulta stavební}}}\\
\vspace*{1em}
\textsf{\textsc{\noun{\Large katedra fyziky}}}\vspace{15mm}


%%% Aby vložení loga  správně fungovalo, je třeba mít soubor lev.png nahraný v pracovním adresáři,
%%% tj. v adresáři, kde se nachází překládaný zdrojový soubor. 
\includegraphics[width=0.3\textwidth]{obrazky/lev}\vspace{15mm}


\textsf{\huge BAKALÁŘSKÁ/DIPLOMOVÁ PRÁCE}{\huge \par}

\vspace{15mm}


\textsf{\LARGE Název práce}{\LARGE \par}

\vspace{10mm}


\end{center} 

\vspace*{\fill}


\vspace{10mm}


\begin{description}
\item [{{\large Autor:}}] \noindent \textsf{\large Jméno autora}{\large \par}
\item [{{\large Vedoucí~práce:}}] \noindent \textsf{\large Doc. RNDr.
Vítězslav Vydra, CSc.}{\large \hfill{}}\textsf{\large Praha, 2008}{\large{}
% doplňte rok vzniku vaší bakalářské práce
}{\large \par}
\end{description}
\clearpage{}

{\small \thispagestyle{plain}\addcontentsline{toc}{chapter}{Abstrakt} }{\small \par}

\noindent {\small ~\vfill{}
}{\small \par}

\begin{description}
\item [{{\small Název~práce:}}] \noindent {\small Název bakalářské práce}{\small \par}
\item [{{\small Autor:}}] \noindent {\small Jméno autora}{\small \par}
\item [{{\small Katedra~(ústav):}}] \noindent Kate{\small dra fyziky}{\small \par}
\item [{{\small Vedoucí~bakalářské~práce:}}] \noindent Doc. RNDr. Vítězslav
Vydra, CSc.
\item [{{\small e-mail~vedoucího:}}] \noindent {\small vydra@fsv.cvut.cz}\\
{\small \par}
\item [{{\small Abstrakt}}] \noindent {\small V předložené práci studujeme...
Uvede se abstrakt v rozsahu 80 až 200 slov. Lorem ipsum dolor sit
amet, consectetuer adipiscing elit. Ut sit amet sem. Mauris nec turpis
ac sem mollis pretium. Suspendisse neque massa, suscipit id, dictum
in, porta at, quam. Nunc suscipit, pede vel elementum pretium, nisl
urna sodales velit, sit amet auctor elit quam id tellus. Nullam sollicitudin.}{\small \par}
\item [{{\small Klíčová~slova:}}] \noindent {\small klíčová slova (3 až
5)}\\
{\small \lyxline{\small}}{\small \par}
\item [{{\small Title:}}] \noindent {\small Název bakalářské práce v angličtině}{\small \par}
\item [{{\small Author:}}] \noindent {\small Jméno autora}{\small \par}
\item [{{\small Department:}}] \noindent {\small Název katedry či ústavu
v angličtině}{\small \par}
\item [{{\small Supervisor:}}] \noindent {\small Jméno s tituly jako v
české verzi, event. pracoviště}{\small \par}
\item [{{\small Supervisor's~e-mail~address:}}] \noindent {\small e-mailová
adresa vedoucího}\\
{\small \par}
\item [{{\small Abstract}}] \noindent {\small In the present work we study
... Uvede se anglický abstrakt v rozsahu 80 až 200 slov. Lorem ipsum
dolor sit amet, consectetuer adipiscing elit. Ut sit amet sem. Mauris
nec turpis ac sem mollis pretium. Suspendisse neque massa, suscipit
id, dictum in, porta at, quam. Nunc suscipit, pede vel elementum pretium,
nisl urna sodales velit, sit amet auctor elit quam id tellus. Nullam
sollicitudin. Donec hendrerit. Aliquam ac nibh. Vivamus mi. Sed felis.
Proin pretium elit in neque. Pellentesque at turpis. Maecenas convallis.
Vestibulum id lectus. }{\small \par}
\item [{{\small Keywords:}}] \noindent {\small klíčová slova (3 až 5) v
angličtině}{\small \par}
\end{description}
\cleardoublepage{}\thispagestyle{empty}~{\small \addcontentsline{toc}{chapter}{Zadání
práce} }{\small \par}



\newpage{}\thispagestyle{empty}~



\newpage{}\thispagestyle{plain}

{\small %\setcounter{page}{3} % nastavení číslování stránek
\ }{\small \par}

\noindent {\small \vfill{}
 % nastavuje dynamické umístění následujícího textu do spodní části stránky
~}{\small \par}

\noindent {\small Prohlašuji, že jsem svou bakalářskou práci napsal(a)
samostatně a výhradně s použitím citovaných pramenů. Souhlasím se
zapůjčováním práce a jejím zveřejňováním.}{\small \par}

{\small \bigskip{}
}\noindent {\small{} V Praze dne \today\hspace{\fill}Jméno Příjmení
+ podpis}\\
{\small{} % doplňte patřičné datum, jméno a příjmení
}{\small \par}

{\small %%%   Výtisk pak na tomto míste nezapomeňte PODEPSAT!
%%%                                         *********
}{\small \par}

\cleardoublepage{}\thispagestyle{empty}{\small \tableofcontents{}% vkládá automaticky generovaný obsah dokumentu
\cleardoublepage{}}{\small \par}

\pagenumbering{arabic}%start arabic pagination from 1 


\chapter{Úvod}

\section{Cíl a metodika práce}

Hlavním cílem práce bude zjištění jestli platforma Vert.x splňuje všechny předpoklady moderní platformy pod kterou lze vyvíjet distribuovanou single-page aplikaci dále jen SPA. 

Hlavním cílem práce bude zjištění jestli se  platforma Vert.x hodí pro vývoj distribuovaných single-page aplikací dále jen SPA. Čtenáři a vytvoření jednoduchého webového mindmap editoru. Jednostránkové webové aplikace pro kolaborativní práci s mindmapami. Na této jednoduché aplikaci bude demonstrován celý proces vývoje webové aplikace pod platformou Vert.x. Při vývoji klientské části bude použit návrhový vzor MVVC. Je nutné uchopit problematiku platformy Vertx v širších souvislostech, proto se práce snaží neopomenout všechny technologie, které s Vertx souvisí, z kterých Vertx vychází nebo které přímo integruje. V teoretické části bude čtenář seznámen s důležitými filozofiemi, které platforma nabízí. A to jak událostmi řízenou architekturou, kterou platforma převzala z dnes již dobře známého frameworku Node.js. Tak především polygnot programováním a jednoduchým konkurenčním modelem. Cílem teoretické části je tedy popsat jednotlivé části platformy a jejich účel či problém, který řeší. V závěru teoretické části bude platforma srovnána s několika významnými frameworky a to v několika důležitých aspektech rychlosti, která je v dnešním světě neustálého růstu počtu zařízení, je to co trápí webové aplikace s desítkami tisíc připojených klientů.

V praktické části bude vytvořen editor pro jednoduchou správu a tvorbu mindmap. Tyto mindmapy bude moct upravovat více uživatelů najednou v reálném čase. Budou popsány a vysvětleny jednotlivé kroky vývoje až po úplné nasazení webové aplikace na jednotlivé pracovní stanice, kde bude prověřena funkčnost distribuovaného provozu aplikace. Pro nasazení aplikace na více pracovních stanic bude použit nástroj konfiguračního managementu Salt Stack.

\section{Postup a předpoklady práce}

Práce předpokládá základní znalost programovacího jazyku Java. Teoretická část se neomezuje pouze na nezbytný popis technologií potřebných k realizaci malé jednostránkové webové aplikaci. Představuje stručný pohled na celou platformu Vert.x. Teoretická část může být použita jako odraz k hlubšímu studiu daných technologií. Pro realizaci webové aplikace budou použity pokročilé techniky, které učiní aplikaci ještě více znovupoužitelnou a škálovatelnou. Tyto techniky budou čtenáři vysvětleny podrobným způsobem s použitím ukázek. Práce předpokládá znalost základní terminologie související s programováním obecně. Méně zažité pojmy budou vysvětleny poznámkou pod čarou.

Při vývoji webové aplikace budou použity následující softwarové technologie:
\begin{itemize}
\item Java Developement Kit 7: soubor základních nástrojů a knihoven pro běh a vývoj Java aplikací.
\item Ubuntu 12.04: operační systém vhodný pro běh Vert.x aplikací
\item Vert.x 2.1M3+: platforma pro vývoj real-time webových aplikací
\item MongoDB: dokumentové orientovaná NoSQL databáze
\item AngularJS: client side framework pro snadný a efektivní vývoj jednostránkových webových aplikací
\item D3.js: framework pro práci s grafy
\end{itemize}

\include{sablona}

\include{uvod_LyX}


\chapter[Závěr]{Závěr}

Práce představila unikátní filosofii a principy frameworku Vert.x. Práce obsahuje také srovnání platformy s jejím nejčastěji zmiňovaným protikandidátem Node.js. V testu výkonů se ukázalo, že framework Node.js nemůže Vert.x konkurovat. A to bez ohledu na jazyk v kterém byly testy implementovány. Srovnání možností ukázalo, že platforma Vert.x toho může nabídnout mnohem více než její předchůdce. 

V praktické části se podařilo vytvořit webovou aplikaci, která splňuje všechny aspekty moderní webové aplikace. Především pak komunikace v reálném čase bez náročných implementací či použití mnoha služeb a nástrojů. V aplikaci je možné jednoduchým a intuitivním způsobem přidávat, přejmenovávat a odebírat její jednotlivé body. Pokud má stejnou myšlenkovou mapu otevřeno více lidí, okamžitě vidí změny, ostatních uživatelů. Aplikace používá volně šiřitelný software, který je ve většině případů špičkové úrovně. % Tento postup dovoluje, s relativně nízkými náklady, řešit velmi komplikované problémy. 
Možnosti pro vylepšení aplikace jsou na straně funkcionální. Bylo by vhodné rozšířit aplikaci o možnost přihlášení a správy pouze svých myšlenkových map nebo případné sdílení jednotlivých map s ostatními uživateli.

Z práce vyplývá, že Vert.x je vysoce modifikovatelný webový framework založený na komunikaci v reálném čase napříč všemi částmi aplikace. Vysoká modularita a otevřenost platformy Vert.x přináší značné výhody pro vývoj webových aplikací, především s dalšími nástroji usnadňujícími vývoj MVC nebo MVVM aplikací, například AngularJS. Již od počátku si kladl za cíl zjednodušit dosavadní možnosti vývoje a představit tak alternativu ke standardním nástrojům vývoje webových aplikací. Je to právě jednoduchost, univerzálnost a komplexnost řešení této platformy, které zlákalo společnost RedHat, která adoptovala tuto platformu. V současné době se velmi progresivně rozšiřuje celý ekosystém okolo Vert.x novými nástroji a možnostmi. 

Díky originálnímu spojení několika klíčových komponent přišla platforma s možností jednoduchého škálování napříč servery. Knihovna Hazelcast představuje klíčovou komponentu pro horizontální škálování. Do již běžícího clusteru lze přidávat nové servery. V režimu HA, lze zajistit vysokou dostupnost na míru celé aplikaci bez nutnosti běhu dalších služeb a pracné konfigurace.

\section{Budoucnost projektu}

Tim Fox hlavní vedoucí projektu představil plán\cite{plan} pro budoucí rozvoj platformy. Nově tak bude šifrovaná veškerá komunikace na Event busu. API bude definováno pomocí vysoce abstraktního programovacího jazyka díky čemuž bude možné generovat API v jiných programovacích jazycích. Zveřejnění jednoduché protokolu pro napojení na Event Bus což de v současné době pouze přes WebSocket a SockJS most. Objevit by se měla také nativní podpora pro Android a IoS.

\section{Možnosti dalšího výzkumu}

Tak rozsáhlé téma jako jsou distribuované webové aplikace rozhodně nelze podrobně popsat v rámci 
jedné bakalářské práce. Na tuto práci proto mohou navazovat kolegové z fakulty či jiných 
vysokých škol. V závěru pro ně přináším dva zajímavá témata, na které již v této práci 
nezbyl prostor a rozhodně si zaslouží podrobnější analýzu.
 
\subsection{Distribuované výpočty}

V dnešní době Big Data\footnote{velká data} je zapotřebí tyto data efektivně a rychle zpracovávat. 

\subsection{Srovnání}

Není


\begin{thebibliography}{10}
\bibitem{Thesis-templates}\emph{\LyX{} and \LaTeX{} Thesis Themplates}
{[}online]. {[}cit. 2008-09-28]. Dostupný z WWW: \url{http://www.thesis-template.com/}

\bibitem{Diplomka-v-LaTeXu}Pele,\emph{ Diplomka v \LaTeX{}u} {[}online].
{[}cit. 2008-09-28]. Dostupný z WWW: \url{pele.gzk.cz/node/37}

\bibitem{Jirkovy-stranky}Roubal, Jiří\emph{. Jirkovy stránky}~{[}online].
{[}cit. 2008-09-28]. Dostupný z WWW: \url{dce.felk.cvut.cz/roubal/}

\bibitem{Lyx-cesky}Vydra, Vítězslav.\emph{ Počeštění \LyX{}u}~{[}online].
2008 {[}cit. 2008-09-28]. Dostupný z WWW: \url{people.fsv.cvut.cz/~vydra/lyxcesky.htm}

\bibitem{Tex-Gyre}\emph{Písmo \TeX-Gyre}~{[}online]. {[}cit. 2008-09-28].
Dostupný z WWW: \url{www.gust.org.pl/projects/e-foundry/tex-gyre}

\bibitem{NeuesKapitel}\emph{Neues Kapitel-Layout}~{[}online]. {[}cit.
2008-09-28]. Dostupný z WWW: \url{www.thesis-template.de/archives/5#more-5}

\bibitem{Vavreckova}Vavrečková, Šárka. \emph{Úprava dokumentů}~{[}online].
{[}cit. 2008-09-28]. Dostupný z WWW:\url{axpsu.fpf.slu.cz/~vav10ui/obsahy/dipl/typografie.pdf}

\bibitem{diplPraceSLU}\emph{Zdroje informací pro diplomové práce,
SLU}~{[}online]. {[}cit. 2008-09-28]. Dostupný z WWW: \url{axpsu.fpf.slu.cz/~vav10ui/obsahy/dipl/typodipl.html}

\bibitem{V-cem}\emph{V čem napsat diplomovou práci}~{[}online].
{[}cit. 2008-09-28]. Dostupný z WWW: \url{www.student.cvut.cz/cwut/index.php/Diplomová_práce#V_.C4.8Dem_napsat_diplomovou_pr.C3.A1ci}

\bibitem{Menousek}Menoušek, Jiří. \emph{Jak (ne)napsat diplomovou
a dizertační práci}~{[}online]. {[}cit. 2008-09-28]. Dostupný z WWW:
\url{www.csmo.cz/other/dizert.php}

\bibitem{Polach}Polách, Eduard. \emph{Pravidla sazby diplomových
prací}~{[}online]. {[}cit. 2008-09-28]. Dostupný z WWW: \url{home.pf.jcu.cz/~edpo/pravidla/pravidla.html}

\bibitem{Citace}Farkašová, Blanka, Krčál, Martin. \emph{Projekt bibliografické
citace}~{[}online]. {[}cit. 2008-09-28]. Dostupný z WWW: \url{www.citace.com}.

\end{thebibliography}
~\\
Tento seznam literatury byl vytvořen přímo v Lyxu pomocí stylu ,,Bibliografie{}``
a generátoru citací~\cite{Citace}. Pořadí citací je takové, jak
je sami napíšeme.

\bibliographystyle{csplainnat}
\bibliography{bakalarka}
~\\
~\\
Tento seznam literatury byl vytvořen pomocí \BibTeX u s použitím
stylu \texttt{csplainnat}. Citace jsou automaticky seřazeny podle
abecedy.

\addcontentsline{toc}{chapter}{Literatura} 

\cleardoublepage{}

\appendix
\pagenumbering{Roman}\addcontentsline{toc}{part}{Přílohy}\thispagestyle{empty}  \renewcommand{\appendixname}{P\v{r}iloha}%%přílohy, číslování římskými


\part*{Přílohy}

\listoffigures

\listoftables

\lstlistoflistings


\end{document}
