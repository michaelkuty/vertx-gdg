%% LyX 1.5.5 created this file.  For more info, see http://www.lyx.org/.
%% Do not edit unless you really know what you are doing.
\documentclass[a4paper,czech,czech,openright,cleardoubleempty,BCOR10mm,DIV11]{scrreprt}
\usepackage[T1]{fontenc}
\usepackage[utf8]{inputenc}
\usepackage{array}
\usepackage{longtable}
\usepackage{varioref}
\usepackage{wrapfig}
\usepackage{fancybox}
\usepackage{calc}
\usepackage{framed}
\usepackage{url}
\usepackage{graphicx}
\usepackage{placeins} %floatbarrier \FloatBarrier
%\usepackage{listing}
\makeatletter

%%%%%%%%%%%%%%%%%%%%%%%%%%%%%% LyX specific LaTeX coěmmands.
\providecommand{\LyX}{L\kern-.1667em\lower.25em\hbox{Y}\kern-.125emX\@}
\newcommand{\lyxline}[1][1pt]{%
  \par\noindent%
  \rule[.5ex]{\linewidth}{#1}\par}
\newcommand{\noun}[1]{\textsc{#1}}
%% Special footnote code from the package 'stblftnt.sty'
%% Author: Robin Fairbairns -- Last revised Dec 13 1996
\let\SF@@footnote\footnote
\def\footnote{\ifx\protect\@typeset@protect
    \expandafter\SF@@footnote
  \else
    \expandafter\SF@gobble@opt
  \fi
}

\renewcommand{\baselinestretch}{1.2} %radkovani s

\expandafter\def\csname SF@gobble@opt \endcsname{\@ifnextchar[%]
  \SF@gobble@twobracket
  \@gobble
}
\edef\SF@gobble@opt{\noexpand\protect
  \expandafter\noexpand\csname SF@gobble@opt \endcsname}
\def\SF@gobble@twobracket[#1]#2{}
%% Because html converters don't know tabularnewline
\providecommand{\tabularnewline}{\\}

%%%%%%%%%%%%%%%%%%%%%%%%%%%%%% Textclass specific LaTeX commands.
\newenvironment{lyxcode}
{\begin{list}{}{
\setlength{\rightmargin}{\leftmargin}
\setlength{\listparindent}{0pt}% needed for AMS classes
\raggedright
\setlength{\itemsep}{0pt}
\setlength{\parsep}{0pt}
\normalfont\ttfamily}%
 \item[]}
{\end{list}}

%%%%%%%%%%%%%%%%%%%%%%%%%%%%%% User specified LaTeX commands.
%<-------------------------------společná nastavení------------------------------>
\usepackage[czech]{babel}%počeštění názvů (Obsah, Kapitola, Literatura atp.)
\usepackage[]{hyperref} %odkazy v  pdf jsou klikací s barevnými rámečky
\usepackage[numbers,sort&compress]{natbib} %balíček pro citace literatury  
\usepackage{hypernat}%interakce mezi hyperref a natbib
\newcommand{\BibTeX}{{\sc Bib}\TeX}%BibTeX logo
\hypersetup{   % Nastavení polí PDF dokumentu 
pdftitle={Platforma Vert.x},%   
pdfauthor={Michael Kutý},%  
pdfsubject={},%   
pdfkeywords={\v{s}ablona,LaTeX,LyX}%                             
}
\usepackage{multicol}




%<-----------------------------volání stylů----------------------------------------->
% (znak % je označení komentáře: co je za ním, není aktivní)
%<------------------------------------písmo----------------------------------------->
%\usepackage{packages/bc-latinmodern}
%\usepackage{packages/bc-times}
\usepackage{packages/bc-palatino}
%\usepackage{packages/bc-iwona}
%\usepackage{packages/bc-helvetika}


%<------------------------------záhlaví stránek------------------------------------>
%\usepackage{packages/bc-headings}
\usepackage{packages/bc-fancyhdr}

%<------------------------------hlavičky kapitol------------------------------------>
%\usepackage{packages/bc-neueskapitel}
%\usepackage{packages/bc-fancychap}

\makeatother

\usepackage{babel}

%java code block%

\usepackage{listing}
\usepackage{listings}
\usepackage{color}

\definecolor{dkgreen}{rgb}{0,0.6,0}
\definecolor{gray}{rgb}{0.5,0.5,0.5}
\definecolor{mauve}{rgb}{0.58,0,0.82}

\renewcommand*{\lstlistingname}{Ukázka kódu} %prejmenovani lstlisting
\renewcommand*{\lstlistlistingname}{Seznam ukázek kódu}

% syntax highlight pro jazyk Java %
\lstset{
  %frame=r,
  language=Java,
  aboveskip=3mm,
  belowskip=3mm,
  xleftmargin=0.2mm,
  showstringspaces=false,
  columns=flexible,
  basicstyle={\small\ttfamily},
  numbers=none,
  numberstyle=\tiny\color{gray},
  keywordstyle=\color{blue},
  commentstyle=\color{dkgreen},
  stringstyle=\color{mauve},
  breaklines=true,
  breakatwhitespace=true,
  tabsize=3,
    inputencoding=utf8,
    extendedchars=true,
    literate=%
    {á}{{\'a}}1
    {č}{{\v{c}}}1
    {ď}{{\v{d}}}1
    {é}{{\'e}}1
    {ě}{{\v{e}}}1
    {í}{{\'i}}1
    {ň}{{\v{n}}}1
    {ó}{{\'o}}1
    {ř}{{\v{r}}}1
    {š}{{\v{s}}}1
    {ť}{{\v{t}}}1
    {ú}{{\'u}}1
    {ů}{{\r{u}}}1
    {ý}{{\'y}}1
    {ž}{{\v{z}}}1
    {Á}{{\'A}}1
    {Č}{{\v{C}}}1
    {Ď}{{\v{D}}}1
    {É}{{\'E}}1
    {Ě}{{\v{E}}}1
    {Í}{{\'I}}1
    {Ň}{{\v{N}}}1
    {Ó}{{\'O}}1
    {Ř}{{\v{R}}}1
    {Š}{{\v{S}}}1
    {Ť}{{\v{T}}}1
    {Ú}{{\'U}}1
    {Ů}{{\r{U}}}1
    {Ý}{{\'Y}}1
    {Ž}{{\v{Z}}}1
}

\begin{document}
%~\thispagestyle{empty}{\small ~\vfill{}
%}{\small \par}

%~\thispagestyle{empty}\vfill{}
%Tato stránka je tzv. protititul a je graficky součástí titulní stránky.
%Nechte ji prázdnou, nebo na ni umístěte vhodnou fotografii či ilustraci.

\cleardoublepage{}~\thispagestyle{empty}\begin{center}\pagenumbering{roman}\vspace{10mm}


\textsf{\textsc{\noun{\LARGE Univerzita Hradec Králové}}}\\
\vspace{0.5em}
\textsc{\noun{\LARGE Fakulta informatiky a managementu}}\\
\vspace*{1em}
\textsf{\textsc{\noun{\Large katedra informatiky a kvantitativních metod }}}

\vspace{15mm}

\includegraphics[width=0.4\textwidth]{logo/uhk}

\vspace{15mm}


\textsf{\huge BAKALÁŘSKÁ PRÁCE}{\huge \par}

\vspace{15mm}


\textsf{\LARGE Vert.x jako platforma pro webové aplikace}{\LARGE \par}

\vspace{10mm}


\end{center} 

\vspace*{\fill}


\vspace{10mm}


\begin{description}
\item [{{\large Autor:}}] \noindent \textsf{\large Michael Kutý}{\large \par}
\item [{{\large Vedoucí~práce:}}] \noindent \textsf{\large doc. Ing. Filip Malý, Ph.D.}{\large \hfill{}}\textsf{\large Hradec Králové, 2014}{\large{}
% doplňte rok vzniku vaší bakalářské práce
}{\large \par}
\end{description}
\clearpage{}

%{\small \thispagestyle{plain}\addcontentsline{toc}{chapter}{Abstrakt} }{\small \par}

\newpage{}\thispagestyle{plain}

{\small %\setcounter{page}{3} % nastavení číslování stránek
\ }{\small \par}

\noindent {\small \vfill{}
 % nastavuje dynamické umístění následujícího textu do spodní části stránky
~}{\small \par}

\subsubsection{Prohlášení}

\noindent {\small Prohlašuji, že jsem bakalářskou práci vypracoval samostatně a uvedl jsem všechny použité prameny a literaturu.}{\small \par}

{\small \bigskip{}
}\noindent {\small{} V Kroměříži dne \today\hspace{\fill}Michael Kutý}\\
{\small{} % doplňte patřičné datum, jméno a příjmení
}{\small \par}

\clearpage{}

\newpage{}\thispagestyle{plain}

{\small %\setcounter{page}{3} % nastavení číslování stránek
\ }{\small \par}

\noindent {\small \vfill{}
 % nastavuje dynamické umístění následujícího textu do spodní části stránky
~}{\small \par}

\subsubsection{Poděkování}

\noindent {\small Rád bych zde poděkoval doc. Ing. Filipu Malému, Ph.D. za odborné vedení práce, podnětné rady a čas, který mi věnoval. \newpage{}}{\small \par}

\clearpage{}

\newpage{}\thispagestyle{plain}

{\small %\setcounter{page}{3} % nastavení číslování stránek
\ }{\small \par}

\noindent {\small \vfill{}
 % nastavuje dynamické umístění následujícího textu do spodní části stránky
~}{\small \par}

\subsubsection{Anotace}

Bakalářská práce se zaměřuje na problematiku vývoje distribuovaných webových aplikací. Cílem je představit základní principy platformy na teoretické úrovni a následně 
je ověřit na praktickém příkladu. Práce je členěna na dvě částí. První část se věnuje obecně celé platformě, druhá část demonstruje teoretické znalosti na praktickém příkladu webové aplikace a ukazuje další možnosti rozšíření. Aplikace se nasadí do dvou referenčních instalací. První do prostředí VirtualBox a druhá v prostředí laboratoře CEPSOS při UHK. Text shrnuje základní možnosti frameworku Vert.x pro začínající uživatele, kteří by rádi použili tento framework pro vývoj webových aplikací.

\subsubsection{Annotation}

EN verze

\cleardoublepage{}

%\thispagestyle{empty}~{\small \addcontentsline{toc}{chapter}{Zadání
%práce} }{\small \par}



\newpage{}\thispagestyle{empty}~


{\small %%%   Výtisk pak na tomto míste nezapomeňte PODEPSAT!
%%%                                         *********
}{\small \par}

\cleardoublepage{}\thispagestyle{empty}{\small 
%\setcounter{secnumdepth}{3}
%\setcounter{tocdepth}{2}%hloubla obsahu
\tableofcontents{}% vkládá automaticky generovaný obsah dokumentu
\cleardoublepage{}}{\small \par}

\pagenumbering{arabic}%start arabic pagination from 1 


\chapter{Úvod}

\section{Cíl a metodika práce}

Hlavním cílem práce bude zjištění jestli platforma Vert.x splňuje všechny předpoklady moderní platformy pod kterou lze vyvíjet distribuovanou single-page aplikaci dále jen SPA. 

Hlavním cílem práce bude zjištění jestli se  platforma Vert.x hodí pro vývoj distribuovaných single-page aplikací dále jen SPA. Čtenáři a vytvoření jednoduchého webového mindmap editoru. Jednostránkové webové aplikace pro kolaborativní práci s mindmapami. Na této jednoduché aplikaci bude demonstrován celý proces vývoje webové aplikace pod platformou Vert.x. Při vývoji klientské části bude použit návrhový vzor MVVC. Je nutné uchopit problematiku platformy Vertx v širších souvislostech, proto se práce snaží neopomenout všechny technologie, které s Vertx souvisí, z kterých Vertx vychází nebo které přímo integruje. V teoretické části bude čtenář seznámen s důležitými filozofiemi, které platforma nabízí. A to jak událostmi řízenou architekturou, kterou platforma převzala z dnes již dobře známého frameworku Node.js. Tak především polygnot programováním a jednoduchým konkurenčním modelem. Cílem teoretické části je tedy popsat jednotlivé části platformy a jejich účel či problém, který řeší. V závěru teoretické části bude platforma srovnána s několika významnými frameworky a to v několika důležitých aspektech rychlosti, která je v dnešním světě neustálého růstu počtu zařízení, je to co trápí webové aplikace s desítkami tisíc připojených klientů.

V praktické části bude vytvořen editor pro jednoduchou správu a tvorbu mindmap. Tyto mindmapy bude moct upravovat více uživatelů najednou v reálném čase. Budou popsány a vysvětleny jednotlivé kroky vývoje až po úplné nasazení webové aplikace na jednotlivé pracovní stanice, kde bude prověřena funkčnost distribuovaného provozu aplikace. Pro nasazení aplikace na více pracovních stanic bude použit nástroj konfiguračního managementu Salt Stack.

\section{Postup a předpoklady práce}

Práce předpokládá základní znalost programovacího jazyku Java. Teoretická část se neomezuje pouze na nezbytný popis technologií potřebných k realizaci malé jednostránkové webové aplikaci. Představuje stručný pohled na celou platformu Vert.x. Teoretická část může být použita jako odraz k hlubšímu studiu daných technologií. Pro realizaci webové aplikace budou použity pokročilé techniky, které učiní aplikaci ještě více znovupoužitelnou a škálovatelnou. Tyto techniky budou čtenáři vysvětleny podrobným způsobem s použitím ukázek. Práce předpokládá znalost základní terminologie související s programováním obecně. Méně zažité pojmy budou vysvětleny poznámkou pod čarou.

Při vývoji webové aplikace budou použity následující softwarové technologie:
\begin{itemize}
\item Java Developement Kit 7: soubor základních nástrojů a knihoven pro běh a vývoj Java aplikací.
\item Ubuntu 12.04: operační systém vhodný pro běh Vert.x aplikací
\item Vert.x 2.1M3+: platforma pro vývoj real-time webových aplikací
\item MongoDB: dokumentové orientovaná NoSQL databáze
\item AngularJS: client side framework pro snadný a efektivní vývoj jednostránkových webových aplikací
\item D3.js: framework pro práci s grafy
\end{itemize}

%\pagenumbering{arabic}%start arabic pagination from 1 

\chapter{Platforma Vert.x}

Dnešním trendem internetu jsou real-time kolaborativní aplikace, které drasticky změnily potřeby programátorů, na jednotlivé nástroje. Programátor tak má možnost zvolit si z velké řádky nástrojů mezi než patří například Node.js, Akka či ruby EventMachine. Problémem těchto jinak časem a komunitou prověřených platforem může být fakt, že jsou úzce spjaté s konkretním programovacím jazykem či velmi náročná integrace do již stávájící aplikace.

Vert.x je projekt vycházející z Node.js, který jako první framework, pokořil v roce 2010 C10K\footnote{C10K problém řeší otázku: „Jak je možné obsloužit deset tisíc klientů za pomocí jednoho serveru, a to s co možná nejnižším zatížením serveru} problém. Platforma Vert.x má velice podobné API\footnote{Application Programming Interface} jako Node.js. Obě platformy poskytují kompletně asynchronní API. Jak již název napovídá Node.js je napsán v JavaScriptu, zatím co Vert.x je implementován v Javě. Vert.x ale nění pouhá reimplementace Node.js do jazyka Java. Platforma má svou vlastní unikátní filozofii, která je diametrálně odlišná od Node.js.

%Problém těchto jinak časem a komunitou ověřených platforem je fakt. Obě zmíněné platformy jsou napsány v dynamicky kompilovaném jazyku, což pro jádro stabilní aplikace přináší povinnost psát jak testy integrační, které testují funkčnost celého systému, tak i unit testy. 
%I když bude aplikace z větší části pokrytá testy, mohou se objevit problémy v podobě nečekaných pádů za běhu aplikace. To může být způsobeno například voláním neexistující metody či přiřazení proměnné do jiného typu než je ona sama. Toto bylo jedním z důvodů pro implementaci nového řešení v jazyce Java. Tento jazyk přináší platformě velkou stabilitu, rozšiřitelnost a zázemí v podobě tisícovek stabilních knihoven. Vert.x může být použit jako plnohodnotné řešení pro celou aplikaci nebo nasazen jako dílčí část architektury jiného řešení.

\section{Historie}

Začátek vývoje projektu Vert.x je datován do roku 2011. Tedy rok poté co spatřil světlo světa framework Node.js a za pouhý rok si vydobyl své místo u komunity, která si jej velmi oblíbila. Pravděpodobně největší motivací pro vývoj nové platformy podobné Node.js byla právě oblíbenost Node.js. 

Hlavním autorem platformy byl a je Tim Fox, který v době začátku vývoje platformy pracoval ve společnosti VMWare. Tato společnost si vzápětí nárokovala všechny zásluhy Tima Foxe na Vert.x platformu. Právníci společnosti vydaly výzvu, ve které požadovali mimo jiné doménu, veškerý zdrojový kód a účet Tima Foxe na Githubu. Z toho důvodu Tim Fox odešel od společnosti v roce 2012. V témže roce projevila o platformu zájem firma RedHat, která nabídla Timovi pracovní místo, absolutně volnou ruku ve vývoji a vedení projektu\citep{whoControlVertx}. 

Po několika debatách jak s představiteli společnosti RedHat tak i komunitou došel Tim Fox k názoru, že nejlepší pro budoucí zdravý rozvoj platformy bude přesunutí celé platformy pod nadaci Eclipse Foundation, k čemuž došlo na konci roku 2013. V dnešní době se platforma těší velkému vývoji, který čítá desítky  pravidelných přispěvatelů mezi něž patří mimo Tima například také Norman Maurer, který patří mezi přední inženýry vyvíjející framework Netty.io, který zodpovídá za integraci Netty frameworku do Vert.x platformy. 

Na tomto místě by bylo vhodné uvést, že platforma Vert.x letos vyhrála prestižní cenu "Most Innovative Java Technology" v soutěži JAX Innovation awards\citep{JAX}.

\section{Architektura}

Na obrázku \ref{fig:vertxArchitectureDiagram} jsou znázorněny dvě nezávislé Vert.x instance, které spolu komunikují pomocí zpráv. V levé části je blíže zobrazena jedna Vert.x instance, která bude blíže rozebrána v následujících kapitolách.

\begin{figure}
\begin{centering}
\includegraphics[width	=1\textwidth]{obrazky/vertx-architecture-diagram}
\par\end{centering}
\caption{Architektura Vert.x \emph{Jaehong Kim} \label{fig:vertxArchitectureDiagram}}
\end{figure}

\subsection{Jádro}

Velikost samotného jádra aplikace nepřekračuje 10Mb kódu v jazyce java. V současné verzi je jádro platformy koherentní, dobře čitelné a poskytuje stabilní API. Lze jej následně rozšířit o novou funkčnost dokompilovaním balíčků, které lze naleznout v oficiálním repositáři. Pravděpodobnou inspirací byl již zmíněný Node.js respektive NPM\footnote{Node package manager} u kterého se takováto forma vývoje velice oblíbila. Od doby vzniku této platformy vzniklo nespočet rozšíření, které udělaly z Node.js silný násroj pro rychlý vývoj webových aplikací. 
Klíčové jsou aspekty jako událostmi řízené programování a neblokující asynchronní model. Událostmi řízené programování je podle Tomáše Pitnera\cite{javaProgramovani} základním principem tvorby aplikací s GUI(Graphical user interface). Netýká se však pouze GUI, je to obecnější pojem označující typ asynchronního programování, kdy je:
tok programu řízen událostmi;
události nastávají obvykle určitou uživatelskou akcí: klik či pohyb myši, stisk tlačítka
událostmi řízené aplikace musí být většinou programovány jako vícevláknové (i když spouštění vláken obvykle explicitně programovat nemusíme)
Asynchronní někdy také paralélní model je přímo závislý na způsobu implementace samotným programovacím jazykem. Základním pojmem je zde proces, který je vnímán jako jedna instance programu, který je plánován pro nezávislé vykonávání. Naproti tomu Vlákno\footnote{Označuje v informatice odlehčený proces, pomocí něhož se snižuje režie operačního systému při změně kontextu, které je nutné pro zajištění multitaskingu} je posloupnost po sobě jdoucích událostí.(vlákno). V dřívější době nebylo potřeba rozlišovat proces a vlákno, protože proces se dále v aplikaci nedělil. Vytvoření vlákna je poměrně drahá a pomalá operace. Což se často obchází vytvořením zásoby uspaných vláken dopředu s nějakým managementem, co vlákna přidává a ubírá dle potřeby. Základním principem Vert.x a jemu podobných frameworků je jedno hlavní vlákno, obvykle pro každý procesor jedno a jednotlivé úlohy co při běhu aplikace vznikají si řídí sám.

%respektive Node.js je tedy jedno hlavní vlákno, které si dle potřeby vytváří vlákna další a řídí tak . V jedné aplikace tedy může běžet několik vláken. Vlákno je zde bráno jako základní plánovací jednotka pro běh na procesoru. 
Existují dva druhy asynchronního modelu (multitaskingu):
multiprocesorový: o běh, tvorbu a režii vláken se stará operační systém
multivláknový: o běh, tvorbu a režii vláken se stará aplikace a předává je operačnímu systému
Podle Lažanského\cite{vlaknaCvut} je sdílení paměti důsledkem nižší režie při přepínání (přepnutí vláken je výrazně rychlejší), obdobně i vytváření a rušení vlákna a samozřejmě i úspora paměti.
Jak již bylo zmíněno jádro Vert.x je implementováno v jazyce Java a pro Vert.x je tedy důležité, jak moc je dobrá implementace paralélního modelu v jazyce JAVA. Zde se dostáváme k jedinému požadavku pro běh Vert.x instancí a to je přítomnost Java development Kitu ve verzi 1.7 a novější. Tato verze přinesla nespočet vylepšení, pro jejichž výpis zde není místo. Došlo také na přepsání či úpravy v několika zásadních třídách z balíčku java.util.concurrent\footnote{Knihovna pro práci s multitaskingem}.
\begin{description}
\item[ExecutorService]{z balíčku java.util.concurrent}
\item[CyclicBarrier\footnote{Synchronizační bariéra. Využitelná pro konstantní skupinu vláken, které mají přistupovat ke stejné proměnné. Třída zajištuje, že na sebe vlákna musí čekat při přístupu k proměnným. (Cyklická, protože jakmile se uvolní první vlákno jede to samé od znova)}]{z balíčku java.util.concurrent}
\item[CountDownLatch]{z balíčku java.util.concurrent}
\item[File]{z balíčku java.nio}
\item[Vylepšený ClassLoader]{lepší odolnost vůči deadlockům\footnote{ Odborný výraz pro situaci, kdy úspěšné dokončení první akce je podmíněno předchozím dokončením druhé akce, přičemž druhá akce může být dokončena až po dokončení první akce.}}
\end{description}
\emph{Více o java.concurrent\cite{javaChangelog}}

Ed Gardoh v roce 2011 ve svém jednoduchém testu\cite{serialTest} prověřil práci s paralelizací úkonů. Z jeho testů vyplývá, že Java 1.7 je až o 40\% rychlejší při práci s vlákny díky nové metodě Fork/Join\footnote{http://www.oracle.com/technetwork/articles/java/fork-join-422606.html}.

\subsection{API}\label{sub:API}

Vert.x poskytuje malou sadu metod, kterou lze volat na přímo z jednotlivých Verticlů.
Funkcionalitu platformy lze jednoduše rozšířit pomocí modulů, které po zveřejnění do centrálního repozitáře může využívat kdokoliv a pomáhá tak znovu použitelnosti kódu. Samotné jádro Vert.x je tak velice malé a kompaktní. Vert.x API se dělí na \emph{Základní API} a \emph{Kontainer API}.

\subsubsection{Základní API}\label{sub:coreAPI}

Základní API, které Vert.x poskytuje programátorovi je poněkud strohé a obdobné jako u frameworku Node.js. Platforma tak poskytuje stabilní základ, který se v praxi neobejde bez modulů o kterých pojednává kapitola \ref{sub:moduly}.

\begin{itemize}
\item{TCP/SSL server/klient}
\item{HTTP/HTTPS server/klient}
\item{Websockets server/klient, SockJS}
\item{Distribuovaný Event Bus}
\item{Časovače}
\item{Práce s buffery}
\item{Přístup k souborovému systému}
\item{Přístup ke konfiguraci}
\end{itemize}

\subsubsection{Kontainer API}

Díky této části API může programátor řídit spouštění a vypínání nových modulů a verticlů za běhu aplikace. V praxi jsme tak schopní škálovat aplikaci za běhu či měnit funkcionalitu celé aplikace aniž by to někdo mohl zaregistrovat. Tuto API můžeme také volat přímo z příkazové řádky dále jen CLI\footnote{Command Line Interface}.

\begin{itemize}
\item{Nasazení a zrušení nasazení Verticlů}
\item{Nasazení a zrušení nasazení Modulů}
\item{Získání konfigurace jednotlivých Verticlů}
\item{Logování}
\end{itemize}

\subsection{Multi-reactor pattern}

Základ jádra je postaven na tzv. Multi-reactor pattern\cite{eventLoops}, který vychází z Reactor patternu\cite{reactorPattern}, ten lze charakterizovat několika body:

\begin{itemize}
\item{aplikace je řízena událostmi}
\item{na události se registrují handlery}
\item{vlákno zpracovává události a spouští registrované handlery}
\item{toto vlákno nesmí být blokováno\footnote{pokud dojde k zablkování hlavního vlákna dojde k zablokování celé aplikace např.\emph{Thread.sleep(), a další z java.util.concurrent }}}
\end{itemize}

Multi-reactor pattern\cite{eventLoops} se od Reactor patternu liší pouze tím, že může mít více hlavních vláken. Tím přináší Vert.x možnost pohodlně škálovat instance na více procesorových jader. Takovému vláknu, se ve Vert.x komunitě říká \emph{Event Loop}. V komunitách Nginx nebo Node.js se ovšem setkáme spíše s pojmem \emph{Run Loop}. Nevýhoda tohoto modelu je, že nikdy nesmí dojít k blokování hlavního vlákna a také fakt, že platforma Node.js poskytovala jenom jedno vlákno, které šlo škálovat na jednotlivé procesory. Jak je vidět z obrázku \vref{fig:instance} Vert.x platforma poskytuje více hlavních vláken, zpravidla však jedno hlavní vlákno na jeden procesor. Toho lze snadno docílit pomocí \emph{Runtime.getRuntime().availableProcessors()} o kterém se dozvíte více v kapitole \ref{sub:Scaling}. Na obrázku \vref{fig:instance4} pak lze vidět situaci čtyř hlavních vláken na čtyři procesorové jádra.
\vref
Příklady blokujících volání:
\begin{itemize}
\item{tradiční API (JDBC, externí systémy)}
\item{dlouhotrvající operace (generování apod.)}
\end{itemize}

\subsubsection{Hybridní model vláken}\label{sub:hybrid}

Platforma Vert.x přišla s inovací v oblasti hlavních vláken a to takovou, že k hlavním \emph{Event loops} přidala další sadu vláken \emph{Background thread pool}, které jsou vyčleněny z hlavní architektury a poskytující samostatnou kapitolu pro škálování aplikace. To lze ostatně vidět na obrázku \vref{fig:vertxArchitectureDiagram}. Díky tomu, lze psát specializované moduly nebo verticle tzv. \emph{workery} pro blokující volání či dlouhotrvající operace aniž by nějak omezovaly běh celé aplikace. Více o \emph{workerech} v \ref{sub:moduly}

\subsection{Vert.x instance}

Verticle běží v jedné Vert.x instanci \vref{fig:instance}. Každá Vert.x instance běží ve vlastním JVM instanci. V jedné Vert.x instanci může najednou běžet X Vertclů. Na jednom fyzickém stroji může běžet více Vert.x instancí případně v cluster módu i na více fyzických strojích.

\begin{figure}
\begin{centering}
\includegraphics[scale=0.5]{obrazky/instance}
\par\end{centering}
\caption{Vert.x instance \label{fig:instance}}
\end{figure}

\begin{figure}
\begin{centering}
\includegraphics[scale=0.5]{obrazky/instance4}
\par\end{centering}
\caption{Vert.x instance \emph{vertx run HelloWord -instances 4} \label{fig:instance4}}
\end{figure}

\subsubsection{Verticle}

Základní jednotka vývoje a nasazení. Verticle může být skript nebo třída například v jazyce Java. Verticle lze spouštět samostatně\footnote{vertx run Verticle.js} v praxi se ovšem využívají pouze moduly, které obsahují zpravidla více Verticles popřípadě worker Verticles.

\begin{itemize}
\item nejmenší spustitelná jednotka
\item třída / skript
\item vykonává neblokující operace
\item konkurence - single-threaded\footnote{běží vždy pouze v jednom vlákně (odpadá synchronizace, zámky, ...), izolace (vlastní classloader)}
\item přístup ke Core API\ref{sub:coreAPI}, registrace handlerů, deploy dalších verticlů
\end{itemize}

Spuštění verticle programově
\begin{lstlisting}
JsonObject config = new JsonObject();
config.putString("foo", "wibble");
config.putBoolean("bar", false);
container.deployVerticle("foo.ChildVerticle", config);
\end{lstlisting}

Spuštění verticle z příkazové řádky
\begin{lstlisting}
vertx run foo.js -conf myconf.json
\end{lstlisting}

\subsubsection{Moduly} \label{sub:moduly}

Moduly poskytují větší míru zapouzdření a znovupoužitelnost funkcionality. V praxi se mohou moduly skládat z více modulů či verticlů a mohou být uloženy v centrálním repozitáři\footnote{http://modulereg.vertx.io/} nebo může být využit jakýkoliv jiný repozitář. Repozitáře v kterých hledá Vert.x při startu instance dostupné moduly lze definovat v hlavní konfiguraci Vert.x.
Každý modul musí mít svůj deskriptor ve formátu JSON\footnote{JSON (JavaScript Object Notation) je odlehčený formát pro výměnu dat. Je jednoduše čitelný i zapisovatelný člověkem a snadno analyzovatelný i generovatelný strojově.}, tento deskriptor musí být v kořenovém adresáři modulu a může vypadat například takto. \emph{toto je poze základní výčet parametrů všechny lze nalézt v dokumentaci Vert.x}

\begin{lstlisting}
{
  "main": "EchoServer.java",
  "worker": true,
  "includes": "io.vertx~some-module~1.1",
  "auto-redeploy": true
}
\end{lstlisting}

Typy modulů lze rozdělit do dvou základních skupin, které lze dál rozdělit podle typu určení modulu. 

\begin{description}
\item[spustitelné]{mají definovanou main třídu v deskriptoru, takovéto moduly je pak možné spustit jako samostatné jednotky pomocí parametru \emph{runmod nebo programově deployModule} }
\item[nespustitelné]{modul nemá specifikovanou main třídu a lze jej použít v jiném modulu použitím parametru \emph{includes}}
\end{description}

Jak bylo řečeno v \ref{sub:hybrid} Vert.x instance má dvě sady vláken. Parametrem \emph{worker} v deskriptoru modulu, lze říci Vert.x jádru aby spustil modul v \emph{background worker poolu}. Parametr \emph{auto-redeploy} mluví sám za sebe.

Spuštění modulu programově v jazyce Java
\begin{lstlisting}
container.deployModule("io.vertx~mod-mailer~2.0.0-beta1", JSONconfig);
\end{lstlisting}

Spuštění modulu z příkazové řádky
\begin{lstlisting}
vertx runmod com.mycompany~my-mod~1.0 -conf config.json
\end{lstlisting}

\subsubsection{Worker Verticle}

\subsection{Event Bus}

Nervový systém celého Vert.x. Cílem EventBusu je zpozdředkování komunikace mezi jednotlivými komponentami platformy. 
Nespornou výhodou je fakt, že lze takovouto komunikaci přemostit ke klientovi na straně webového prolížeče.

Základní typy komunikace:
\begin{itemize}
\item{Point to Point}
\item{Publish/Subscribe}
\end{itemize}

typy zpráv:
\begin{itemize}
\item{String}
\item{primitivní typy (int, long, short, float double, ..)}
\item{org.vertx.java.core.json.JsonObject}
\item{org.vertx.java.core.buffer.Buffer}
\end{itemize}

Toto jsou pouze základní typy zpráv, které Vert.x podporuje v základu. Není ale vůbec problém výčet stávájících typů rozšířit(doimplementovat). Například modul bson.vertx.eventbus\footnote{https://github.com/pmlopes/mod-bson-io} rozšíří aplikaci o možnost používat mnohem komplexnější typy zpráv. Mezi doporučené se ovšem řadí JSON, protože je jednoduše serializovatelný mezi jednotlivými programovacími jazyky.
\begin{itemize}
\item{java.util.UUID}
\item{java.util.List}
\item{java.util.Map}
\item{java.util.Date}
\item{java.util.regex.Pattern}
\item{java.sql.Timestamp}
\end{itemize}

\subsection{Hazelcast}

Jednou z nejdůležitějších architektonických součástí Vert.x je knihovna Hazelcast\footnote{okolo 2.6MB kódu v jazyce Java, In-Memory Data Grid (IMDG)}, Hlavní výhody In-memory data grid\cite{inMemoryDataGrid} lze podle Ki Sun Song sumarizovat:
\begin{itemize}
\item{Data jsou distribuovaná a uložená na více servrech }
\item{Datový model je většinou objektově orientovaný a ne-relační}
\item{Každý server pracuje v aktivním režimu}
\item{Dle potřeby lze přidávat a odebírat servery}
\end{itemize}

Hazelcast lze využít v několika rolích:
\begin{itemize}
\item{In-memory NoSQL\footnote{databázový koncept, ve kterém datové úložiště i zpracování dat používají jiné prostředky než tabulková schémata tradiční relační databáze}}
\item{Caching\footnote{specializovaný typ paměti pro krátkodobé ukládání}}
\item{Data grid}
\item{Messaging}
\item{Application Scaling}
\item{Clustering}
\end{itemize}

Hazelcast je tedy typ distribuovaného úložiště, které běží jako embedded a lze díky němu distribuovat celou aplikaci. Hazelcast API je využíváno přes API Vert.x. Když je Vert.x spuštěn, Hazelcast je spuštěn v embedded\footnote{Hazelcast server je spuštěn jádrem Vert.x} módu. 
Jako nejčastější příklad bývá uváděno ukládání uživatelské session\footnote{Session v protokolu HTTP dává webovému serveru možnost uložit si libovolné (většinou však ne příliš obsáhlé) informace o uživatelích, kteří k němu přistupují, a to o každém zvlášť. Protokol HTTP ze svého principu (a způsobu komunikace stylem požadavek - odpověď) postrádá kontext o jednotlivých klientech, a právě session ho webovým aplikacím dokáže dát.} Hazelcast tedy usnadní práci v situaci, kdy budeme potřebovat uložit uživatelskou session například pro eshop. Mohli bychom využít využit externí RDBMS\footnote{Databázový server, který spravuje databáze, komunikaci s klienty (lokálními nebo vzdálenými), vstupy a výstupy dat a jejich integritu.} díky, kterému by jsme dosáhli stejného výsledku. Hazelcast nám ovšem zaručí replikování mezi jednotlivými servery, fail-over S využitím embedded Hazelcast ovšem odpadá nezbytná režie a monitoring, nemluvě o serverových prostředcích.

Proto ty, kteří potřebují ukládat uživatelské session pro E-commercy či chat-servery toho mohou jednoduše dosáhnout skrz konfiguraci samotného Vert.x.


\section{Test}
Ed Gardoh v roce 2011 provedl test\cite{serialTest} pro porovnání paralelizace\footnote{Paralelizace procesů se skládá z rozložení jednoho velkého úkonu do několika menších úkolů, které mohou běžet paralelně.Výsledkem je provedení jednoho úkolu nebo procesu za pomocí více než jednoho procesoru nebo procesorů "Paralelní zpracování", nesmí být zaměňováno se souběžností.} v Javě 1.6 a 1.7.
Hlavní myšlenkou je aby testovací třída simulovala úkol, který jako první volá vzdálenou službu a čeká sekundu na výzvu k návratu(spánek) a pak simuluje nějaké zpracování s výsledkem, jako je formátování řetězce.
\vref{fig:serialtestI} je vidět synchronní běh serializační třídy v Javě 1.6. Z \vref{fig:serialCputestI} je pak vidět využití potenciálů jednotlivých procesorů.
Výsledek není žádné překvapení 50 úkolů s 1 sekundovým spánkem a spojováním řetězce trvalo něco málo přes 65 sekund. 
Cílem jeho testu mělo být porovnání paralelizování úkonů. Výsledky testu ukázaly zlepšení až o 75\%. Z obrázků 1-4 zřetelně plyne, že nová Java je, pro single-thread\footnote{jedno vláknový} model aplikace ta nejlepší volba.

Jetnotlivé testy prokázaly, že za takovým rapidním zrychlením stojí metody Fork/Join. Při vhodném škálování bylo zrychlení až o 75\%. Z testů ovšem vyplívá také fakt, že při neúměrném počtu hlavních vláken na počet procesorů to má negativní dopady. Jedním z dopadů je 100\% vytížení a jednotlivých jader. Při vhodném určení počtu vláken, je vidět rapidní urychlení asynchronní paralelizace. Node.js i Vert.x však poskytuji informace o celkovém počtu fyzických jader procesoru a ta je tedy snadné určení optimálního počtu vláken pro ideální výsledky.(Více na?asi vysvětlit)

\begin{figure}
\begin{centering}
\includegraphics[width=1\textwidth]{obrazky/serial_test_r1}
\par\end{centering}
\caption{První test běhu serializační třídy \label{fig:serialtestI}}
\end{figure}

\begin{figure}
\begin{centering}
\includegraphics[width=1\textwidth]{obrazky/serial_cpu_1}
\par\end{centering}
\caption{Využití jednotlivých procesorů při běhu \label{fig:serialCputestI}}
\end{figure}

\begin{figure}
\begin{centering}
\includegraphics[width=1\textwidth]{obrazky/executor_test_r1}
\par\end{centering}
\caption{První test běhu serializační třídy \label{fig:executorTestI}}

\end{figure}

\begin{figure}
\begin{centering}
\includegraphics[width=1\textwidth]{obrazky/executor_cpu_1}
\par\end{centering}
\caption{První test běhu serializační třídy \label{fig:executorCpuTestI}}

\end{figure}


\pagenumbering{arabic}%start arabic pagination from 1 

\chapter{Praktická část}

popis

\section{Návrh}

test

\section{Realizace}

test


\chapter[Závěr]{Závěr}

Práce představila unikátní filosofii a principy frameworku Vert.x. Práce obsahuje také srovnání platformy s jejím nejčastěji zmiňovaným protikandidátem Node.js. V testu výkonů se ukázalo, že framework Node.js nemůže Vert.x konkurovat. A to bez ohledu na jazyk v kterém byly testy implementovány. Srovnání možností ukázalo, že platforma Vert.x toho může nabídnout mnohem více než její předchůdce. 

V praktické části se podařilo vytvořit webovou aplikaci, která splňuje všechny aspekty moderní webové aplikace. Především pak komunikace v reálném čase bez náročných implementací či použití mnoha služeb a nástrojů. V aplikaci je možné jednoduchým a intuitivním způsobem přidávat, přejmenovávat a odebírat její jednotlivé body. Pokud má stejnou myšlenkovou mapu otevřeno více lidí, okamžitě vidí změny, ostatních uživatelů. Aplikace používá volně šiřitelný software, který je ve většině případů špičkové úrovně. % Tento postup dovoluje, s relativně nízkými náklady, řešit velmi komplikované problémy. 
Možnosti pro vylepšení aplikace jsou na straně funkcionální. Bylo by vhodné rozšířit aplikaci o možnost přihlášení a správy pouze svých myšlenkových map nebo případné sdílení jednotlivých map s ostatními uživateli.

Z práce vyplývá, že Vert.x je vysoce modifikovatelný webový framework založený na komunikaci v reálném čase napříč všemi částmi aplikace. Vysoká modularita a otevřenost platformy Vert.x přináší značné výhody pro vývoj webových aplikací, především s dalšími nástroji usnadňujícími vývoj MVC nebo MVVM aplikací, například AngularJS. Již od počátku si kladl za cíl zjednodušit dosavadní možnosti vývoje a představit tak alternativu ke standardním nástrojům vývoje webových aplikací. Je to právě jednoduchost, univerzálnost a komplexnost řešení této platformy, které zlákalo společnost RedHat, která adoptovala tuto platformu. V současné době se velmi progresivně rozšiřuje celý ekosystém okolo Vert.x novými nástroji a možnostmi. 

Díky originálnímu spojení několika klíčových komponent přišla platforma s možností jednoduchého škálování napříč servery. Knihovna Hazelcast představuje klíčovou komponentu pro horizontální škálování. Do již běžícího clusteru lze přidávat nové servery. V režimu HA, lze zajistit vysokou dostupnost na míru celé aplikaci bez nutnosti běhu dalších služeb a pracné konfigurace.

\section{Budoucnost projektu}

Tim Fox hlavní vedoucí projektu představil plán\cite{plan} pro budoucí rozvoj platformy. Nově tak bude šifrovaná veškerá komunikace na Event busu. API bude definováno pomocí vysoce abstraktního programovacího jazyka díky čemuž bude možné generovat API v jiných programovacích jazycích. Zveřejnění jednoduché protokolu pro napojení na Event Bus což de v současné době pouze přes WebSocket a SockJS most. Objevit by se měla také nativní podpora pro Android a IoS.

\section{Možnosti dalšího výzkumu}

Tak rozsáhlé téma jako jsou distribuované webové aplikace rozhodně nelze podrobně popsat v rámci 
jedné bakalářské práce. Na tuto práci proto mohou navazovat kolegové z fakulty či jiných 
vysokých škol. V závěru pro ně přináším dva zajímavá témata, na které již v této práci 
nezbyl prostor a rozhodně si zaslouží podrobnější analýzu.
 
\subsection{Distribuované výpočty}

V dnešní době Big Data\footnote{velká data} je zapotřebí tyto data efektivně a rychle zpracovávat. 

\subsection{Srovnání}

Není


\begin{thebibliography}{10}

\bibitem{whoControlVertx}Phipps, Simon \emph{Who controls Vert.x: Red Hat, VMware, or neither?}
{[}online]. {[}cit. 2014-02-16]. Dostupný z WWW: \url{http://www.infoworld.com/d/open-source-software/who-controls-vertx-red-hat-vmware-or-neither-210549}

\bibitem{JAX}Kamali, Masoud \emph{The Winners of the JAX Innovation Awards 2014}
{[}online]. {[}cit. 2014-03-20]. Dostupný z WWW: \url{http://jax.de/awards2014/}

\bibitem{serialTest}Gardoh, Ed \emph{Parallel Processing and Multi-Core Utilization with Java}
{[}online]. {[}cit. 2014-03-22]. Dostupný z WWW: \url{http://embarcaderos.net/2011/01/23/parallel-processing-and-multi-core-utilization-with-java/}

\bibitem{reactorPattern}Merta, Zdeněk\emph{Vert.x jOpenSpace 2013} {[}online].
{[}cit. 2014-03-22]. Dostupný z WWW: \url{http://jopenspace.cz/2013/presentations/zdenek-merta-vert.x.pdf}

\bibitem{forkJoin}Ponge, Julien \emph{ Fork and Join: Java Can Excel at Painless Parallel Programming Too! } {[}online].
{[}cit. 2014-03-22]. Dostupný z WWW: \url{http://www.oracle.com/technetwork/articles/java/fork-join-422606.html}

\bibitem{javaChangelog}\emph{ Package java.util.concurrent Description} {[}online].
{[}cit. 2014-03-22]. Dostupný z WWW: \url{http://docs.oracle.com/javase/7/docs/api/java/util/concurrent/package-summary.html#package_description}

\bibitem{inMemoryDataGrid}Sun Song, Ki \emph{ Understanding Vert.x Architecture - Part II } {[}online].
{[}cit. 2014-03-22]. Dostupný z WWW: \url{http://www.cubrid.org/blog/dev-platform/introduction-to-in-memory-data-grid-main-features/}

\bibitem{vertxArchitectureDiagram}Jaehong, Kim\emph{ Introduction to In-Memory Data Grid: Main Features } {[}online].
{[}cit. 2014-03-22]. Dostupný z WWW: \url{http://www.cubrid.org/blog/dev-platform/understanding-vertx-architecture-part-2/}

\bibitem{javaProgramovani}Pitner, Tomáš\emph{ Programování v jazyce Java } {[}online].
{[}cit. 2014-04-10]. Dostupný z WWW: \url{http://www.fi.muni.cz/~tomp/slides/pb162/printable.html}

\bibitem{vlaknaCvut}Lažanský, J.\emph{ Procesy a vlákna } {[}online].
{[}cit. 2014-04-15]. Dostupný z WWW: \url{http://labe.felk.cvut.cz/vyuka/A4B33OSS/Tema-03-ProcesyVlakna.pdf}

\bibitem{eventLoops}Fox, Tim\emph{ Event loops } {[}online].
{[}cit. 2014-04-15]. Dostupný z WWW: \url{http://vertx.io/manual.html#event-loops}

\bibitem{session}Kosek, Jiří\emph{ Session proměnné } {[}online].
{[}cit. 2014-04-15]. Dostupný z WWW: \url{http://www.kosek.cz/clanky/php4/session.html}

\bibitem{mq}Janssen, Cory\emph{ Message Queue } {[}online].
{[}cit. 2014-04-22]. Dostupný z WWW: \url{http://www.techopedia.com/definition/25971/message-queue}

\bibitem{benchmark}Froemke, Dina\emph{ Framework Benchmarks Round 8 } {[}online].
{[}cit. 2014-04-22]. Dostupný z WWW: \url{http://www.techempower.com/blog/2013/12/17/framework-benchmarks-round-8/}

\bibitem{benchmarkTim}Fox, Tim\emph{ Vert.x vs node.js simple HTTP benchmarks } {[}online].
{[}cit. 2014-06-22]. Dostupný z WWW: \url{http://vertxproject.wordpress.com/2012/05/09/vert-x-vs-node-js-simple-http-benchmarks/}

\bibitem{scaling}Osuszek, Lukasz\emph{ Distributed Architecture of Enterprise Information Systems } {[}online].
{[}cit. 2014-07-31]. Dostupný z WWW \url{http://www.soainstitute.org/resources/articles/distributed-architecture-enterprise-information-systems}

\bibitem{d3js}Bostock, Mike\emph{ Collapsible Tree } {[}online].
{[}cit. 2014-07-31]. Dostupný z WWW \url{http://bl.ocks.org/mbostock/4339083}

\bibitem{ubuntu}Canonical Ltd.\emph{ Ubuntu Server Edition } {[}online].
{[}cit. 2014-08-01]. Dostupný z WWW \url{http://www.ubuntu.com/server}

\bibitem{mongodb}Williams, Alex\emph{ MongoDB Raises 150M For NoSQL Database Technology With Salesforce Joining As Investor } {[}online].
{[}cit. 2014-08-02]. Dostupný z WWW \url{http://techcrunch.com/2013/10/04/mongodb-raises-150m-for-nosql-database-technology-with-salesforce-joining-as-investor/}

\bibitem{sockjs}Rauch, Guillermo\emph{ WebSocket emulation } {[}online].
{[}cit. 2014-08-02]. Dostupný z WWW \url{https://github.com/sockjs}

\bibitem{webSockets}Malý, Martin\emph{ Web sockets } {[}online].
{[}cit. 2014-08-02]. Dostupný z WWW \url{http://www.zdrojak.cz/clanky/web-sockets/}

\bibitem{javaPKG}Oracle\emph{ Naming a Package } {[}online].
{[}cit. 2014-08-02]. Dostupný z WWW \url{http://docs.oracle.com/javase/tutorial/java/package/namingpkgs.html}



\end{thebibliography}

\addcontentsline{toc}{chapter}{Literatura} 

\cleardoublepage{}

\appendix
\pagenumbering{Roman}\addcontentsline{toc}{part}{Přílohy}\thispagestyle{empty}  \renewcommand{\appendixname}{P\v{r}iloha}%%přílohy, číslování římskými


\part*{Přílohy}

\listoffigures

\listoftables

\lstlistoflistings


\end{document}
