
\chapter[Závěr]{Závěr}

V práci byla představena platforma Vert.x jako nástroj pro distribuované webové aplikace. Byla popsána unikátní filozofie a terminologie této platformy.

V praktické části se podařilo vytvořit webovou aplikaci, která splňuje všechny aspekty moderní webové aplikace. Především pak komunikace v reálném čase bez náročných implementací či použití mnoha služeb a nástrojů. V aplikace je možné jednoduchým a intuitivním způsobem přidávat a edebírat jednotlivé body. Pokud má stejnou myšlenkovou mapu otevřeno více lidí, okamžitě vidí všechny změny, ostatních klientů. Aplikace používá volně šiřitelný software, který je ve většině případů špičkové úrovně. % Tento postup dovoluje, s relativně nízkými náklady, řešit velmi komplikované problémy. 
Možnosti pro vylepšení aplikace jsou jak na straně vizuální tak na straně funkcionální. Bylo by vhodné rozšířit aplikaci o možnost přihlášení a správy pouze svých myšlenkových či případné sdílení jednotlivých map s ostatními uživateli.

 
Z práce vyplývá, že se platforma Vert.x hodí pro vývoj webových aplikací, výhradně pak za použití s dalšími nástroji usnadňující vývoj MVC případně MVVM aplikací, například Spring frameworkem.

\section{Možnosti dalšího výzkumu}

Tak rozsáhlé téma jako jsou distribuované webové aplikace rozhodně nelze podrobně popsat v rámci 
jedné bakalářské práce. Na tuto práci proto mohou navazovat kolegové z fakulty či jiných 
vysokých škol. V závěru pro ně přináším dva zajímavá témata, na které již v této práci 
nezbyl prostor a rozhodně si zaslouží podrobnější analýzu.
 
\subsection{Distribuované výpočty}

sad
 
8.1.2 Cloudové databáze 
Jde o databáze, které jsou primárně navrženy pro použití v cloudu. Cloud computing a 
NoSQL databáze je jedno z diskutovaných témat dnešní doby a do budoucna význam těchto 
databází nepochybně poroste. Jde třeba o databáze Cloudant, Cloudbase či Amazon 
DynamoDb. 
