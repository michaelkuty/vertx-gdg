\pagenumbering{arabic}%start arabic pagination from 1 

\chapter{Úvod}

V současné době existuje nespočet frameworků\footnote{Cílem frameworku je převzetí typických problémů dané oblasti, čímž se usnadní vývoj tak, aby se návrháři a vývojáři mohli soustředit pouze na své zadání} pro vývoj webových aplikací ve spoustě programovacích jazycích. Vývojář tak nemá vůbec lehké, vybrat ten správný nástroj, který by mu zaručil, že se jeho aplikace dostane na hranici možností, které mu daný nástroj poskytuje. 

Většina webových aplikací ovšem dříve nebo později narazí na na problematiku škálování, kdy je třeba rozložit aplikaci na vice serverů ať už pro zajištění vysoké dostupnosti nebo co nejnižší odezvy. Dnes také není nic neobvyklého, že aplikaci najednou začnou navštěvovat tisíce klientů za minutu a rázem se tak může stát, že z jinak rychlé aplikace se stane často padající aplikace s nepřiměřenou odezvou.

Právě proto, jsem se rozhodl k hlubšímu zkoumání v dané oblasti webových aplikací. V první části bakalářské práce je popsána architektura a jednotlivé technologie, které mě motivovali k hlubšímu studiu platformy Vert.x. V hlavní části práce následuje návrh a vlastní implementace jednostránkové aplikace. V závěru je pak shrnutí kladů a záporů platformy. 

\section{Cíl a metodika práce}

Hlavním cílem práce bude zjištění zda-li se  platforma Vert.x hodí pro vývoj distribuovaných jednostránkových aplikací dále jen SPA. Vytvoření jednoduchého webového editoru myšlenkových map dále jen mindmap. %Jednostránkové webové aplikace pro kolaborativní práci s mindmapami. % 
Na této jednoduché aplikaci bude demonstrován proces vývoje webové aplikace pod platformou Vert.x. Při vývoji klientské části bude použit návrhový vzor MVVC.

Je nutné uchopit problematiku platformy Vert.x v širších souvislostech, proto se práce snaží neopomenout všechny technologie, které s Vert.x souvisí, z kterých Vert.x vychází nebo které přímo integruje. V teoretické části bude čtenář seznámen s důležitými filozofiemi, které platforma nabízí. A to jak událostmi řízenou architekturou, kterou platforma převzala z dnes již dobře známého frameworku Node.js. Tak především polygnot programování s jednoduchým konkurenčním modelem a možnost sdílet data mezi jednotlivými vlákny bez nutnosti zámků.

Cílem teoretické části je tedy popsat jednotlivé části platformy a jejich účel či problém, který řeší. V závěru teoretické části bude platforma srovnána s již zmíněným nástrojem Node.js\footnote{Serverový framework, postavený na modelu událostmi řízeného programování} to v několika důležitých aspektech rychlosti, která je v dnešním světě neustálého růstu počtu zařízení, to co trápí webové aplikace s desítkami tisíc dlouho trvajících připojení.

V praktické části bude vytvořen editor pro jednoduchou správu a tvorbu mindmap. Tyto mindmapy bude moct upravovat více uživatelů najednou v reálném čase. Budou popsány a vysvětleny jednotlivé kroky vývoje až po úplné nasazení webové aplikace na jednotlivé servery, kde bude prověřena funkčnost distribuovaného provozu aplikace. Pro nasazení aplikace na více serverů bude použit nástroj konfiguračního managementu Salt Stack.

\section{Postup a předpoklady práce}

Práce předpokládá základní znalost programovacího jazyku Java a JavaScript. Teoretická část se neomezuje pouze na nezbytný popis technologií potřebných k realizaci malé jednostránkové webové aplikace. Představuje stručný pohled na celou platformu Vert.x. Teoretická část může být použita jako odraz k hlubšímu studiu daných technologií. Pro realizaci webové aplikace budou použity pokročilé techniky, které učiní aplikaci ještě více znovupoužitelnou a škálovatelnou. Tyto techniky budou čtenáři vysvětleny podrobným způsobem s použitím ukázek. Práce předpokládá znalost základní terminologie související s programováním obecně. Méně zažité pojmy budou vysvětleny poznámkou pod čarou.

Při vývoji webové aplikace budou použity následující softwarové technologie:
\begin{itemize}
\item Java Developement Kit 7: soubor základních nástrojů a knihoven pro běh a vývoj Java aplikací.
\item Ubuntu 12.04: operační systém vhodný pro běh Vert.x aplikací
\item Vert.x 2.1M3+: platforma pro vývoj real-time webových aplikací
\item MongoDB: dokumentové orientovaná NoSQL databáze
\item AngularJS: client side framework pro snadný a efektivní vývoj jednostránkových webových aplikací
\item D3.js: framework pro práci s grafy
\end{itemize}