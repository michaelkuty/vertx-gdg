\pagenumbering{arabic}%start arabic pagination from 1 


\chapter{Úvod}

\section{Cíl a metodika práce}

Hlavním cílem práce bude zjištění jestli platforma Vert.x splňuje všechny předpoklady moderní platformy pod kterou lze vyvíjet distribuovanou single-page aplikaci dále jen SPA. 

Hlavním cílem práce bude zjištění jestli se  platforma Vert.x hodí pro vývoj distribuovaných single-page aplikací dále jen SPA. Čtenáři a vytvoření jednoduchého webového mindmap editoru. Jednostránkové webové aplikace pro kolaborativní práci s mindmapami. Na této jednoduché aplikaci bude demonstrován celý proces vývoje webové aplikace pod platformou Vert.x. Při vývoji klientské části bude použit návrhový vzor MVVC. Je nutné uchopit problematiku platformy Vertx v širších souvislostech, proto se práce snaží neopomenout všechny technologie, které s Vertx souvisí, z kterých Vertx vychází nebo které přímo integruje. V teoretické části bude čtenář seznámen s důležitými filozofiemi, které platforma nabízí. A to jak událostmi řízenou architekturou, kterou platforma převzala z dnes již dobře známého frameworku Node.js. Tak především polygnot programováním a jednoduchým konkurenčním modelem. Cílem teoretické části je tedy popsat jednotlivé části platformy a jejich účel či problém, který řeší. V závěru teoretické části bude platforma srovnána s několika významnými frameworky a to v několika důležitých aspektech rychlosti, která je v dnešním světě neustálého růstu počtu zařízení, je to co trápí webové aplikace s desítkami tisíc připojených klientů.

V praktické části bude vytvořen editor pro jednoduchou správu a tvorbu mindmap. Tyto mindmapy bude moct upravovat více uživatelů najednou v reálném čase. Budou popsány a vysvětleny jednotlivé kroky vývoje až po úplné nasazení webové aplikace na jednotlivé pracovní stanice, kde bude prověřena funkčnost distribuovaného provozu aplikace. Pro nasazení aplikace na více pracovních stanic bude použit nástroj konfiguračního managementu Salt Stack.

\section{Postup a předpoklady práce}

Práce předpokládá základní znalost programovacího jazyku Java. Teoretická část se neomezuje pouze na nezbytný popis technologií potřebných k realizaci malé jednostránkové webové aplikaci. Představuje stručný pohled na celou platformu Vert.x. Teoretická část může být použita jako odraz k hlubšímu studiu daných technologií. Pro realizaci webové aplikace budou použity pokročilé techniky, které učiní aplikaci ještě více znovupoužitelnou a škálovatelnou. Tyto techniky budou čtenáři vysvětleny podrobným způsobem s použitím ukázek. Práce předpokládá znalost základní terminologie související s programováním obecně. Méně zažité pojmy budou vysvětleny poznámkou pod čarou.

Při vývoji webové aplikace budou použity následující softwarové technologie:
\begin{itemize}
\item Java Developement Kit 7: soubor základních nástrojů a knihoven pro běh a vývoj Java aplikací.
\item Ubuntu 12.04: operační systém vhodný pro běh Vert.x aplikací
\item Vert.x 2.1M3+: platforma pro vývoj real-time webových aplikací
\item MongoDB: dokumentové orientovaná NoSQL databáze
\item AngularJS: client side framework pro snadný a efektivní vývoj jednostránkových webových aplikací
\item D3.js: framework pro práci s grafy
\end{itemize}